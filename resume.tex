%%%%%%%%%%%%%%%%%%%%%%%%%%%% Document Setup %%%%%%%%%%%%%%%%%%%%%%%%%%%%
\documentclass[12pt]{article}
\renewcommand{\familydefault}{\sfdefault}
% -----------------------------------------------------------------------
\def \address{214 Preston Court D\\Catonsville, Maryland, 21228}
%\def \address{12885 NW Westlawn Terrace\\Portland, Oregon, 97229}
%\def \address{13717 Chadron Avenue\\Hawthorne, California, 90250}
\def \author{Joe Shields}
\def \email{
		%\href{mailto:shields6@pdx.edu}{shields6@pdx.edu} \\ 
		%\href{mailto:joseph.shields@spacex.com}{joseph.shields@spacex.com} \\ 
		\href{mailto:joedang100@gmail.com}{joedang100@gmail.com} \\
		%\href{mailto:joe.shields@umbc.edu}{joe.shields@umbc.edu} \\
		\href{Joedang.github.io}{Joedang.github.io}
	   }
\def \phnum{(971)-226-9393}
% -----------------------------------------------------------------------
\pdfinfo{ % appears to not work with pdflatex
	/Author (Joe Shields)
	/Title (Joe Shields' Resume)
	/Keywords (mechanical;engineering;physics;composite;interdisciplinary;math)
}
% Layout: Puts the section titles on left side of page
\reversemarginpar

\usepackage[paper=letterpaper,
            marginparwidth=5mm,       % Length of section titles
            marginparsep=1.5mm,       % Space between titles and text
            margin=20mm,              % margins
            %includemp,
			noheadfoot,
			top=10mm,
			pdftex
            ]{geometry}
            
\usepackage{mathtools}            
\usepackage{blindtext}
\usepackage{verbatim}
\usepackage{multicol}

%% More layout: Get rid of indenting throughout entire document
\setlength{\parindent}{0in}

%% This gives us fun enumeration environments. compactitem will be nice.
\usepackage{paralist}

% \usepackage{fontsetname}
\usepackage{fancyhdr}
\pagestyle{fancy}
\pagestyle{empty}      % Uncomment this to get rid of page numbers
% Finally, give us PDF bookmarks
\usepackage{color}
%\usepackage[hidelinks]{hyperref}
\usepackage[colorlinks=true, 
            urlcolor=blue,
            pdfauthor={Joe Shields},
            pdftitle={Joe Shields' Resume},
            pdfsubject={resume},
            pdfkeywords={physics mechanical engineering space composite math interdisciplinary research development}
            ]{hyperref}
%%%%%%%%%%%%%%%%%%%%%%%% End Document Setup %%%%%%%%%%%%%%%%%%%%%%%%%%%%

%%%%%%%%%%%%%%%%%%%%%%%%%%% Helper Commands %%%%%%%%%%%%%%%%%%%%%%%%%%%%
\renewcommand{\section}[2]%
        {\pagebreak[2]\vspace{0.01\baselineskip}
         \phantomsection\addcontentsline{toc}{section}{#1}%
		 \textsc{{#1}}%\rmfamily{\scshape{#1}}
         \hspace{50in}\vspace{2.5mm}\noindent\hrule
        }

% An itemize-style list with lots of space between items
\newenvironment{outerlist}[1][\enskip\textbullet]%
        {\begin{itemize}[#1]}{\end{itemize}%
         \vspace{-.8\baselineskip}}

% An environment IDENTICAL to outerlist that has better pre-list spacing
% when used as the first thing in a \section
\newenvironment{lonelist}[1][\enskip\textbullet]%
        {\vspace{-\baselineskip}\begin{list}{#1}{%
        \setlength{\partopsep}{6pt}%
        \setlength{\topsep}{0pt}}}
        {\end{list}\vspace{-.6\baselineskip}}

% An itemize-style list with little space between items
\newenvironment{innerlist}[1][\enskip\textbullet]%
        {\begin{compactitem}[#1]}{\end{compactitem}
        \vspace{0.5\baselineskip}}

\newcommand{\hlinewidth}{0.01pt}

\newcommand{\jobitem}[3]{%
  \par\vspace{2pt}\noindent
  \makebox[0pt][l]{#1}%
  \makebox[\textwidth][c]{#2}%
  \makebox[0pt][r]{\textbf{#3}}\par\vspace{1pt}}

%%%%%%%%%%%%%%%%%%%%%%%% End Helper Commands %%%%%%%%%%%%%%%%%%%%%%%%%%%

%%%%%%%%%%%%%%%%%%%%%%% Begin Document %%%%%%%%%%%%%%%%%%%%%%%%%%%%%
\begin{document}
\newlength{\rcollength}\setlength{\rcollength}{3.0in}%
\begin{tabular} {
  p{\textwidth/3-\marginparwidth+2mm}
  p{\textwidth/3-\marginparwidth+2mm}
  p{\textwidth/3-\marginparwidth+2mm}}
	\begin{flushleft}
		\address
	\end{flushleft}
	&
	\begin{center}
	\Large{\author}
	\end{center}
	&
	\begin{flushright}
		{\phnum}\\
 	    {\email} 	 
	\end{flushright}
\end{tabular}

% ----------------------------------------------------------------------------
%\vfill
\section{Tools}
\noindent\rule{\textwidth}{\hlinewidth}
    \begin{innerlist}
    \item 3D printers (FDM, SLA, SLS, MJP), laser cutters, mills, lathes, hand tools
    \item PCB routers, soldering (hand, re-flow), oscilloscopes, various microscopes
    \item Composites manufacturing (wet, dry, high/low-temperature), metal working
    \end{innerlist}

% ----------------------------------------------------------------------------
%\vfill
\section{Software}
\noindent\rule{\textwidth}{\hlinewidth}
    \begin{innerlist}
        \item R, MATLAB, C++, Python, Bash, Vim, Git, Jekyll, HTML, CSS, JavaScript
        \item Inventor, SolidWorks, NX, Teamcenter, Onshape, Abaqus, AutoCAD, GIMP, Inkscape
        \item \LaTeX, Microsoft Office, Libre Office, Google Docs, et cetera
        \item Linux, Windows
    \end{innerlist}

% ----------------------------------------------------------------------------
%\vfill
\section{Experience}
\noindent\rule{\textwidth}{\hlinewidth}

\begin{innerlist}
    \item Design Engineer \hfill\textbf{Oct. 2021 -- July 2023}
    \\ \href{https://esi.umbc.edu/}{Earth and Space Institute} and \href{https://www.airphoton.com/}{AirPhoton}
        % given title: Faculty Research Assistant (ESI), Engineer (AirPhoton)
        % first day: 2021-10-18
        % last day: 2023-07-07
        % ESI address: 1000 Hilltop Cir, Baltimore, MD 21250
        % AP address: 1442 S Rolling Rd, Lansdowne, MD 21227
        % 60% time at ESI, 40% time at AP
        % AP supervisor: John Hall
        % ESI supervisor: Dr. Vanderlei Martins
        \subitem Rescued a 20-million-dollar project by applying physical scaling laws
        \subitem Halved the size of three separate systems through clever architecture changes
        \subitem Performed space-claim, keep-out-zone, and tolerance analyses 
        \subitem Designed cameras and optical calibration systems
        \subitem Managed system requirements, interfaces, and performance
        \subitem Designed optics by manually tracing rays in Autodesk Inventor
        \subitem Coordinated with contractors, customers, scientists, and engineers to write system specifications
        \subitem Designed orbital, airborne, and ground-based instruments
        \subitem Solved multi-disciplinary design constraints (mechanical, optical, pneumatic, thermal, etc.)
	\item Mechanical Lead \hfill\textbf{Sep. 2019 -- Oct. 2021} 
    \\ \href{https://www.pdxaerospace.org/}{Portland State Aerospace Society (PSAS)}
		\subitem Mentored student projects and assembled project teams
		\subitem Maintained equipment and lab space
		\subitem Performed FMEA and root-cause analysis
		\subitem Managed interdisciplinary projects among students and professionals
    \item Lab Manager \hfill\textbf{Sep. 2019 -- Oct. 2021}
    \\ \href{https://psu-epl.github.io/}{PSU Electronics Prototyping Lab}
		\subitem Maintained equipment and lab space
		\subitem Trained students on prototyping equipment
		\subitem Ran the lab's parts store
	\item Engineer \hfill\textbf{Mar. 2019 -- Sep. 2019}
    \\ SpaceX 
        % given title: Manufacturing Engineer
        % first day: 2019-03-18
        % last day: 2019-09-16
        % Address: 1 Rocket Rd, Hawthorne, CA 90250
        % Department: Dragon 2, Mechanisms, Manufacturing Engineering
        % Full time, 50+ hours per week
        % Supervisor: Nico Robert, nico.robert@spacex.com, 424-200-0492
		\subitem Supported a wide variety of mechanisms on the 
            \href{https://en.wikipedia.org/wiki/Dragon_2}{human-rated Dragon 2 docking systems}
		\subitem Wrote detailed and intuitive assembly instructions to meet strict quality standards
		\subitem Owned aggressive build schedules and held others accountable to them
		\subitem Solved issues including design errors, part damage, missing parts, and documentation errors
    \pagebreak
    \item R\&D Engineer \hfill\textbf{Sep. 2018 -- Feb. 2019} %remember to add links to these things
    \\ \href{http://pacificdt.com/}{Pacific Diabetes Technologies} 
        % given title: Mechanical Engineer
        % first day: 2018-09-01
        % last day: 2019-02-01
        % Address: 12172 SW Garden Pl, Portland, OR 97223
        % Half time, 20+ hours per week
        % Supervisor: Dr. Robert (Bob) Cargill, bcargill@pacificdt.com, 503-679-6349
        \subitem Prototyped wearable micro-fluidic devices and electronic enclosures
		\subitem Created designs, models, and drawings for patent applications
		\subitem Designed miniaturized assemblies for 3D printing and injection molding
	\item Mechanical Lead \hfill\textbf{Dec. 2015 -- Mar. 2019} 
    \\ \href{https://www.pdxaerospace.org/}{Portland State Aerospace Society (PSAS)}
		\subitem \href{https://github.com/psas/sw-cad-airframe-lv3.0}{Created 
			an open-hardware carbon fiber rocket airframe} for the 
			\href{http://psas.pdx.edu/}{Portland State Aerospace Society}
		\subitem Published and presented 
            \href{http://arc.aiaa.org/doi/pdf/10.2514/6.2016-5365}{a conference paper on the project for AIAA SPACE 2016}
		\subitem Documented design and manufacturing processes to foster institutional knowledge
		\subitem Designed parts using hand calculations, prototypes, computer models, CFD, and CAD
	\item Design Engineer \hfill\textbf{Jan. 2017 -- Mar. 2019}
    \\ \href{https://www.oresat.org/}{OreSat}
		\subitem Coordinated the design of all mechanical subsystems in \href{http://oresat.org/}{Oregon's first satellite}
		\subitem Maintained the \href{https://github.com/oresat/oresat-structure}{top-level SolidWorks assembly} of the satellite
		\subitem Incorporated constraints from NASA, NanoRacks, and OreSat electrical subsystems
		\subitem Worked across engineering disciplines to resolve highly coupled designs
		\subitem Led analysis and design reviews 
%	\item PCC Art Beat competition \hfill\textbf{May 2012}
%       % award date: 
%		\subitem Composed and conducted an original piece for about 30 members
%		\subitem Earned $1^\text{st}$ place in the composition competition
    \item Lab Manager \hfill\textbf{Jan. 2018 -- Mar. 2019, Sep. 2019 -- Oct. 2021}
    \\ \href{https://psu-epl.github.io/}{Electronics Prototyping Lab}
        \subitem Same duties listed above
	\end{innerlist}


% ----------------------------------------------------------------------------
%\vfill
\section{Small Projects}
\noindent\rule{\textwidth}{\hlinewidth}
In addition to the projects below, you can check out the rest of my portfolio at 
\href{https://github.com/Joedang}{github.com/Joedang}.
	\begin{innerlist}
    \item iTopie printer \\
    Modified and built a RepRap 3D printer from parts including a custom laser-cut frame
	\item \href{https://github.com/Joedang/restricted\_three\_body\_problem}{Restricted 3-body simulation} \\
	An R script for investigating the motion of satellites within planet-moon systems
	%\item \href{https://github.com/Joedang/Portfolio/tree/master/projectile}{Ballistic trajectory simulation} \\
	%Realistic scenarios of short-range ballistic motion of various projectiles on different planets, 
    %accounting for buoyancy, drag, centrifugal, and Coriolis effects written in R
	\item \href{https://github.com/Joedang/Portfolio/tree/master/MATLAB\_orbits}{N-body simulation} \\
	Various scenarios involving an arbitrary number of charged massive particles written in MATLAB
	\item OpenFOAM analysis\\
	A model of supersonic flow around a rocket nosecone, used to inform the part's design
	%\item \href{https://github.com/joedang/pats}{PSAS Asset Tracking System} \\
	%Created a specification and front-end in R Shiny for a website to track part maintenance and ownership
	%\item Wearable device enclosure \\
	%Created a 3D printed enclosure for a wearable sensor prototype for \href{https://www.apdm.com/}{APDM} using SolidWorks
	\end{innerlist}

% ----------------------------------------------------------------------------
% \vfill
% \section{Skills and Interests}
% \noindent\rule{\textwidth}{\hlinewidth}
% \begin{multicols}{2}
% 	\begin{innerlist}
% 	\item Classical field theory
% 	\item Mathematical physics
% 	\item Statistics and reliability
% 	\item Thermal and fluid analysis
% 	\item Interdisciplinary research and engineering
% 	\item Music theory
% 	\item Leading small groups
% 	\item Composites manufacturing methods
% 	\end{innerlist}
% \end{multicols}

% ----------------------------------------------------------------------------
%\vfill
% \section{References}
% \noindent\rule{\textwidth}{\hlinewidth}
% 	\begin{innerlist}
% 	\item Supervisors
% 		\subitem Andrew Greenberg -- PSAS director\hfill\href{mailto:adg4@pdx.edu}{adg4@pdx.edu} % 503-708-7711
%         \subitem J. Vanderlei Martins -- principal investigator \hfill 301-828-7471, \href{mailto:martins@umbc.edu}{martins@umbc.edu}
% 		\subitem Eric Russo -- senior engineer \hfill 714-395-8453, \href{mailto:eric.russo@spacex.com}{eric.russo@spacex.com}
% 		\subitem Chris Clark -- EPL director \hfill\href{mailto:cjclark@pdx.edu}{cjclark@pdx.edu} % 503-789-4636
% 		\subitem Erik S\'anchez, PhD -- professor \hfill\href{mailto:esanchez@pdx.edu}{esanchez@pdx.edu}
% 		\subitem Erin Schmidt -- former PSAS mechanical lead \hfill\href{mailto:esch2@pdx.edu}{esch2@pdx.edu}
% 	\item Peers
%         \subitem Mitch Weiss \hfill \href{mailto:mweiss@airphoton.com}{mweiss@airphoton.com}
% 		\subitem Douglas Schmidt \hfill\href{mailto:daschmid@alumni.cmu.edu}{daschmid@alumni.cmu.edu}
% 		\subitem Calvin Young \hfill\href{mailto:youngcal@pdx.edu}{youngcal@pdx.edu}
% 		\subitem Adam Harris \hfill\href{mailto:alegendaryhamster@gmail.com}{alegendaryhamster@gmail.com} % 704-488-0608
% 		\subitem Marie House \hfill\href{mailto:hmarie@pdx.edu}{hmarie@pdx.edu}
% 	\end{innerlist}

% ----------------------------------------------------------------------------
%\vfill
\section{Education}
	\noindent\rule{\textwidth}{\hlinewidth}
	% you can duplicate lines below to include other degree you have
	\begin{innerlist}
	\item Portland State University, 3.65 GPA	\hfill\textbf{Sep. 2013 -- Jun. 2016}
        % start date: 2013-09-30
        % degree date: 2016-06-12
        % last attended: 2016-12-10
        % GPA for last 2 years of PSU undergrad: 3.90
		\subitem \textbf{B.S. Mechanical Engineering}, Maseeh College of Engineering and Computer Science
		\subsubitem Focus: heat and mass transfer
		\subitem\textbf{B.S. Physics}, College of Liberal Arts and Sciences
		\subsubitem Focus: classical mechanics and electromagnetism
		\subitem 
	\item Portland Community College, 3.0 GPA	\hfill\textbf{Sep. 2008 -- Jun. 2010, Sep. 2011 -- Sep. 2013}
        % start date 1:
        % end date 1:
        % start date 2:
        % end date 2:
	\end{innerlist}

% ----------------------------------------------------------------------------
% GRE %
% registration number: 0223155
% general test:
%    date: 2016-10-20
%    verbal:
%       score: 160
%       percentile: 86
%       range: 130 to 170
%    quantitative:
%       score: 164
%       percentile: 84
%       range: 130 to 170
%    writing:
%       score: 3.5
%       percentile: 39
%       range: 0 to 6
% physics test:
%    scaled score: 780
%    percentile: 63

% ----------------------------------------------------------------------------
%\vfill
\pagebreak
\section{Miscellaneous}
	\noindent\rule{\textwidth}{\hlinewidth}
    References can be provided upon request.

    I'm only open to engineering work that is either:
    \begin{innerlist}
    \item fully remote, or
    \item hybrid (meaning no more than 24 in-person hours per week) and located in Cascadia (Oregon or Washington state).
    \end{innerlist}
    Please do not contact me about roles that fail the above criteria.

    I'm most interested in design engineering, ideally optomechanical design (though the sub-field isn't a hard requirement).
    I work very well with scientists and other engineering disciplines (especially electrical engineers).
    I perform best with minimal oversight.
    In my ideal workflow, I spend one to two consecutive days coordinating with team members, 
    and three to four consecutive days hyperfocusing on the tasks identified by that coordination.

% ----------------------------------------------------------------------------
\vfill
\centering rendered \today
\end{document}
%%%%%%%%%%%%%%%%%%%%%%%%%% End RESUME Document %%%%%%%%%%%%%%%%%%%%%%%%%%%%%

% GRE Scores
% verbal reasoning scaled score: 160
% verbal reasoning percent below: 85
% quantitative reasoning scaled score: 164
% quantitative reasoning percent below: 87
% analytic writing scaled score: 3.5
% analytic writing percent below: 42
% physics subject scaled score: 780
% physics subject percent below: 67
